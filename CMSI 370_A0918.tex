\documentclass[11pt]{article}
\usepackage{anysize}
\usepackage{graphicx}
\marginsize{1 in}{1 in}{0.5 in}{6 pt}
\title{CMSI 370-01 Assignment 0918}
\author{Chris Whiting}
%\setlength{\intextsep}{-0.5ex}
\begin{document}
\maketitle

For our experiment, we wanted to test the learnability, efficiency, and satsifaction of HBO GO and Netflix. We ran four tests in order to rate and record each of these metrics. The Learnability of a system is the time it takes to accomplish a task with that system from a subject who is unfamiliar with the system. The Efficiency of a system is the time it takes to accomplish a task with that system from a subject who is familiar with the system. The last metric to be considered while running our tests was Satisfaction, which is the rating at which each subject either enjoyed or hated performing tasks on the system. 
 
\begin{verbatim}
TEST 1: Search Movies/TV Shows
\end{verbatim}
Our first test is to search for a certain movie or tv show and although HBO Go and Netflix have a different selection, we can still perform accurate tests because we are not testing for each sites' selections. We timed each participant on how fast this task could be accomplished. 

\begin{table}[ht]
\caption{Movie/TV Show Results} % title of Table
\centering % used for centering table
\begin{tabular}{|c|c c c c| c |}  %ADD MORE COLUMNS HERE
\hline\hline %inserts double horizontal lines
HBOGO show  &Justin& Mikey&Kevin&Travis&Average  \\ [0.5ex] % ENTER MORE SUBJECTS HERE
%heading
\hline % inserts single horizontal line
Black Swan  & 8.6s & 9.5s&6.1s &11.4s&8.9s    \\ %ADD MORE TIMES HERE
Entourage &9.7s&8.1s&9.0s &21.31s&12.0s  \\
Game of Thrones &6.5s&7.0s&6.9s &5.6s&6.5s \\ 
\hline %inserts single line
Netflix show &Justin& Mikey&Kevin&Travis&Average   \\ [0.5ex]
\hline
Adventureland &5.0s&8.5s&6.5s&6.1s&6.5s   \\
Scary Movie &5.0s&7.3s&6.5s&6.5s& 6.3s    \\
River Monsters &5.0s&6.2s&6.5s&4.8s&5.6s      \\
\hline
\end{tabular}
\label{table:nonlin} % is used to refer this table in the text
\end{table}

\begin{verbatim}
TEST 2: Video Navigation
\end{verbatim}
Video navigation categorizes our second test, where we timed how long each person took to fast forward from the start of a tv show up to around 15 minutes. Also the time it took to navigate to the next episode in line was recorded.

\begin{table}[ht]
\caption{Video Navigation Results} % title of Table
\centering % used for centering table
\begin{tabular}{|c| c c c c |c|} % centered columns (4 columns)  %ADD MORE COLUMNS HERE
\hline\hline %inserts double horizontal lines
HBOGO  &Justin& Mikey&Kevin&Travis&Average   \\ [0.5ex] % ENTER MORE SUBJECTS HERE
%heading
\hline % inserts single horizontal line
GOT 15min FF  &12.2s & 9.9s&7.3s&3.5s&8.2s     \\ %ADD MORE TIMES HERE
GOT (ep6 - ep7) &19.9s&21.8s&26.9s&27.6s&24.1s   \\ 
\hline %inserts single line
Netflix &Justin& Mikey&Kevin&Travis&Average    \\[0.5ex]
\hline
River Monsters 15min FF &6.0s&7.0s&5.2s&4.8s&5.8s \\
River Monsters (ep1 - ep2) &5.7s&6.3s&9.9s&3.9s&6.45s    \\
\hline
\end{tabular}
\label{table:nonlin} % is used to refer this table in the text
\end{table}


\begin{verbatim}






TEST 3: Genre/Filter Search
\end{verbatim}
For this test, we timed how long it took each participant to go from the home page of the website to a web page with only movies of that specific genre. We timed for comedy and romance movies. HBO GO has a dedicated Comedy button from the home page, but most people did not see this. 

\begin{table}[ht]
\caption{Filter Search for Movies Results} % title of Table
\centering % used for centering table
\begin{tabular}{|c|c c c c|c|}   %ADD MORE COLUMNS HERE
\hline\hline 
HBOGO  &Justin& Mikey&Kevin&Travis&Average   \\ [0.5ex] % ENTER MORE SUBJECTS HERE
%heading
\hline 
Comedy  &11.0s & 14.7s&12.9s&7.6s&11.6s  \\
Romance &4.6s&60.0s&8.8s&5.5s&19.7s    \\
\hline
Netflix &Justin& Mikey&Kevin&Travis&Average     \\ [0.5ex]
\hline
Comedy &9.6s&13.6s&9.5s&2.4s&8.8s   \\ %ADD MORE TIMES HERE
Romance &4.1s&N/A&9.7s&2.6s&5.5s  \\ 
\hline %inserts single line
\end{tabular}
\label{table:nonlin} % is used to refer this table in the text
\end{table}

\begin{verbatim}
TEST 4: Cast Member Search
\end{verbatim}
Searching for the web page with the cast and crew of a particular movie or tv show was the last test we timed. We had two movies/tv shows for each website and we timed how long it took each person to navigate from the home page to the page with the cast information.

\begin{table}[ht]
\caption{Filter Search for Movies Results} 				% title of Table
\centering 									% used for centering table
\begin{tabular}{|c|c c c c|c|} % centered columns (4 columns)  %ADD MORE COLUMNS HERE
\hline\hline 									%inserts double horizontal lines
HBOGO  &Justin& Mikey&Kevin&Travis&Average  \\ [0.5ex] 	    % ENTER MORE SUBJECTS HERE
%heading
\hline 									    % inserts single horizontal line
Black Swan  &16.8s & 18.5s&12.4s&15.8s&15.9s          \\			 %ADD MORE TIMES HERE
Hesher &9.5s&7.5s&6.7s&5.7s&7.4s                             \\
\hline
Netflix &Justin& Mikey&Kevin&Travis&Average   \\ [0.5ex] 			% [1ex] adds vertical space
\hline
Adventureland &9.0s&11.6s&10.5s&11.7s&10.7s          \\
True Grit &7.2s&15.1s&7.2s&9.9s&9.9s                         \\
\hline 										%inserts single line
\end{tabular}
\label{table:nonlin} % is used to refer this table in the text
\end{table}

\begin{verbatim}
SATISFACTION
\end{verbatim}
At the end of each test we asked each person to rate the websites from 1 to 10, 10 meaning using that particular site was very satisfying and 1 being not satisfying. Based on Table 5 below, our test subjects favored HBOGO slightly more than Netflix. They said that "HBO GO was easier to navigate" and "HBO GO looked prettier than Netflix". Also, based on the UI of each website, the HBOGO website is less cluttered with icons, advertisements, and miscellaneous buttons, which may have attributed to its higher satisfaction rate.

I feel like HBOGO performed the best, not just because of the data we gathered, but because in general, the interface was much cleaner than that of Netflix. Some people had trouble selecting the movie/tv show in HBOGO at first, because they were unfamiliar with it. After the first try, however, they selected movies/tv shows much faster. In Netflix, people seemed to start off searching for movies faster, but did not improve much. All of the test subjects were familiar with Netflix and only one was familiar with HBO GO. This is because Netflix has a free trial period that is easy to get, while HBO GO requires that the user already have an HBO subscription.

For learnability, the tests that required priority are tests 1 and 2. Test 1 has prioritizes learnability, because even if a user is new to one of these websites, they should learn how to search for a movie and play it, with relative ease. The next time they come back to either Netflix or HBO GO, they would search for shows very fast, because of their high learnability. If a movie streaming site is hard to use, then the user will not use that site, and thus they make no money. Test 2 also prioritizes learnability, once a user figures out how to play, pause, or stop a show, the next time they perform this task, it will go much faster. The results in Table 2 prove this; however, this table also shows how HBO GO's video navigation is hard to learn, compared to that of Netflix, but once figured out, the navigation becomes much faster.

For efficiency, the tests that required priority are tests 3 and 4. The genre or filter search requires efficiency, because the reason a user goes to HBO GO or Netflix is so they can watch a show as quickly as possible. The time it takes to search through genres corresponds with part of the efficiency of the website. Information about a movie or tv show should also be accomplished as fast as possible, and since most movie streaming sites have similar setups, there is little learnability here. Efficiency should be prioritized so that the user has to do little extra work when the show is selected to view the cast and crew of that particular show.

All of the tests also test satisfaction, because the way each website operates as well as how it looks attributes to the overall satisfaction of the user. Even if HBO GO has some interface problems, the simplistic design and high learnability of the website gives higher satisfaction with its users than that of Netflix.

\begin{table}[ht]
\caption{Satisfaction Results} 				% title of Table
\centering 									% used for centering table
\begin{tabular}{|c|c c c c|c|} % centered columns (4 columns)  %ADD MORE COLUMNS HERE
\hline\hline 									%inserts double horizontal lines
   &Justin& Mikey&Kevin&Travis&Average  \\ [0.5ex] 	    % ENTER MORE SUBJECTS HERE
%heading
\hline 									    % inserts single horizontal line
HBO GO  &7.6s & 8.5s&9.0s&8.0s&8.3s          \\			 %ADD MORE TIMES HERE
\hline
Netflix &8.8s&8.0s&7.0s&9.0s&8.2s                             \\ [0.5ex]
\hline
\end{tabular}
\label{table:nonlin} % is used to refer this table in the text
\end{table}

\begin{verbatim}
GUIDELINES
\end{verbatim}
There are no guidelines documents for HBO GO nor Netflix, therefore I will be using the guidelines about 16:3 Ensure that Necessary Information is Displayed and 16:4 Group Related Elements on the usability.gov website and determine whether HBO GO and Netflix comply with these guidelines.

\underline{HBOGO:}
 The first guideline tries to ensure that the website displays the necessary information for the user. HBO GO seems to accomplish this task very well. The simplistic design to the homepage and the homepage displays all of the required categories a user would think of when going to the site. (i.e. movies, series, search bar, etc.) The second guideline is somewhat related to the first guideline, but is more specific. It states that a website should group all related information and functions in order to decrease time spent searching or scanning. HBO GO's homepage groups all the movies in the movies section, tv shows in the series section, etc. The website complies with this guideline very well.

\underline{Netflix:}
For the first guideline, Netflix complies with it reasonable well, however, the homepage required many clicks to get to the necessary information or displays the required information in wierd ways. It tries to speed up selecting a movie by puting related movies to the ones just watched on the homepage, but this just adds more clutter. This website seems to focus more on the medium the movie is watched (i.e. dvd, online, a video queue, etc.) while most people go to Netflix to watch a movie not make these types of choices.
The elements on Netflix's homepage show ways of navigating to what the user wants, however, the website shows related movies and tv shows to the ones the user recently watched, which may or may not be what the user intended to watch. Because of this, Netflix complies with the second guideline reasonably, but not the best.

\underline{Metrics Vs. Guidelines:}
Learnability is the time it takes a user to learn a system and for this project, the time it took a new user to complete a task. Both of the guidelines I am analyzing contribute to this time. Efficiency and Satisfaction are also effected by the guidelines above, because how fast a familiar user completes the same task and the overall user satisfaction with the design of the site are based upon if HBO GO and Netflix were successful in keeping to the guidelines, two of which are stated above.






\end{document}

