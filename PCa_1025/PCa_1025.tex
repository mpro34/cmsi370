\documentclass[11pt]{article}
\usepackage{anysize}
\usepackage{graphicx}
\marginsize{1 in}{1 in}{0.5 in}{6 pt}
\title{ Proficiency Check 1025a}
\author{Chris Whiting}
%\setlength{\intextsep}{-0.5ex}
\begin{document}
\maketitle

 
\section*{Questions}

\begin{enumerate}
\item What, typically, is the weakest usability metric of the menus/forms/dialogs interaction style, and why?

\item Name a feature that can be included in a menus/forms/dialogs user interface that addresses this metric.
\end{enumerate}

\section*{Answers}

\begin{enumerate}
\item 
Memorability is the weakest usability metric of the menus/forms/dialogs interaction style, because these rely on recognition instead of recall. In other words, these interaction styles attempt to make a system that is easily recognizable by new users and old, while the memorability is an afterthought.

\item 
Symbols or icons within menus are examples of a system's memorability. Most people think of the homepage when they see a Home button in a menu, which makes that icon memorable to the user. 
\end{enumerate}

% JD: Take a closer look at your first answer---in many ways, it is actually
%     self-contradictory, and casts some doubt on how well you understand the
%     conceptual material.
%
%     First of all, you stated in your other answer that learnability is the
%     strongest metric of the menus/forms/dialogs interaction style.  If that
%     is the case, then memorability cannot possibly be the weakest metric,
%     because a learnable system is inherently memorable!  Think about it: if a
%     user can figure something out very quickly without having *ever* seen a
%     system, then, after seeing it already, wouldn't that user *also* be able
%     to figure it out quite quickly even after being away for some time?  If a
%     user interface is learnable, then even if a user has completely forgotten
%     it, he/she would be able to figure it out quickly again, thus making it
%     automatically memorable too!
%
%     Thus, memorability cannot be the weakest metric if learnability is the
%     strongest one.
%
%     Now, let's look at your second answer.  Now, *if* memorability were the
%     weakest metric of this interaction style, I would agree that *well-chosen*
%     symbols and icons will improve this metric.  Your answer, taken as is,
%     sort of implies this, so I don't take its omission against you, but take
%     note that in terms of what you actually said, this is missing.
%
%     Based on your answers, your proficiencies remain the same, so no change
%     to your standards achievement report here.

\end{document}
