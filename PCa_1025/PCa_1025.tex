\documentclass[11pt]{article}
\usepackage{anysize}
\usepackage{graphicx}
\marginsize{1 in}{1 in}{0.5 in}{6 pt}
\title{ Proficiency Check 1025a}
\author{Chris Whiting}
%\setlength{\intextsep}{-0.5ex}
\begin{document}
\maketitle

 
\section*{Questions}

\begin{enumerate}
\item What, typically, is the weakest usability metric of the menus/forms/dialogs interaction style, and why?

\item Name a feature that can be included in a menus/forms/dialogs user interface that addresses this metric.
\end{enumerate}

\section*{Answers}

\begin{enumerate}
\item 
Satisfaction, because menus/forms/dialogs are not meant to look or appeal to their users. They are supposed to be easy to learn, memorable, and in some instances efficient. For example, keyboard shortcuts to a menu allows for more efficiency. Making menus that are satisfying may distract the user when they are trying to figure out how to complete a task. 

\item 
Menus/forms/dialogs are supposed to be highly learnable. The new version of Firefox is trying to make their browser more satisfying by simplifying their top menu bar down to a single orange drop down menu called the "Firefox Button". This may look satisfying and sleeker, however the vastly different menu system as compared to Firefox's previous version makes it harder to learn and thus less memorable. Also, they condense all tasks to a single menu instead of having everything spread out on the top of the browser page.
\end{enumerate}

% JD: Take a closer look at your first answer---in many ways, it is actually
%     self-contradictory, and casts some doubt on how well you understand the
%     conceptual material.
%
%     First of all, you stated in your other answer that learnability is the
%     strongest metric of the menus/forms/dialogs interaction style.  If that
%     is the case, then memorability cannot possibly be the weakest metric,
%     because a learnable system is inherently memorable!  Think about it: if a
%     user can figure something out very quickly without having *ever* seen a
%     system, then, after seeing it already, wouldn't that user *also* be able
%     to figure it out quite quickly even after being away for some time?  If a
%     user interface is learnable, then even if a user has completely forgotten
%     it, he/she would be able to figure it out quickly again, thus making it
%     automatically memorable too!
%
%     Thus, memorability cannot be the weakest metric if learnability is the
%     strongest one.
%
%     Now, let's look at your second answer.  Now, *if* memorability were the
%     weakest metric of this interaction style, I would agree that *well-chosen*
%     symbols and icons will improve this metric.  Your answer, taken as is,
%     sort of implies this, so I don't take its omission against you, but take
%     note that in terms of what you actually said, this is missing.
%
%     Based on your answers, your proficiencies remain the same, so no change
%     to your standards achievement report here.

\end{document}
