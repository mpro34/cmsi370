\documentclass[11pt]{article}
\usepackage{anysize}
\usepackage{graphicx}
\marginsize{1 in}{1 in}{0.5 in}{6 pt}
\title{Proficiency Check 1025b}
\author{Chris Whiting}
%\setlength{\intextsep}{-0.5ex}
\begin{document}
\maketitle

 
\section*{Questions}

\begin{enumerate}
\item What, typically, is the strongest usability metric of the menus/forms/dialogs interaction style?
\item What is the cognitive rationale behind why the menus/forms/dialogs interaction style tends to excel at the above usability metric? (Hint: There are two key words here.)
\end{enumerate}

\section*{Answers}

\begin{enumerate}
\item
 Learnability

\item
Menus excel at learnability because these types of systems rely on recognition instead of recall; cognitively an easier task for users to perform.

\end{enumerate}

% JD: You got the strongest metric right, but the reason that you give does not
%     get to the core of things.  Your second answer still gives the *effects*
%     of the cognitive rationale behind the menus/forms/dialogs interaction style,
%     and not the *cause*.  It is the cause that we are looking for, with those
%     two key words.
%
%     Based on these answers, I see no change in proficiencies, so we leave things
%     the same in the standards achievement report.
%
%     Also, one note---this is actually the portion marked as "PCa" in your
%     standards achievement report.  I won't change filenames for now, but I
%     thought I'd mention it so you know which is which.

\end{document}
