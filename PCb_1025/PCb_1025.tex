\documentclass[11pt]{article}
\usepackage{anysize}
\usepackage{graphicx}
\marginsize{1 in}{1 in}{0.5 in}{6 pt}
\title{Proficiency Check 1025b}
\author{Chris Whiting}
%\setlength{\intextsep}{-0.5ex}
\begin{document}
\maketitle

 
\begin{verbatim}
Questions:
1. What, typically, is the strongest usability metric of the menus/forms/dialogs interaction style?
2. What is the cognitive rationale behind why the menus/forms/dialogs interaction style tends to excel at the above usability metric? (Hint: There are two key words here.)
\end{verbatim}

\begin{enumerate}
\item
 Learnability 

\item
Menus excel at learnability because these types of systems are particularly effective for:
\begin{enumerate}
\item
Users with little training
\item
Intermittent use of the system
\item
Users who are unfamiliar with task terminology
\end{enumerate}

\end{enumerate}

\end{document}
