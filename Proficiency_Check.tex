\documentclass[11pt]{article}
\usepackage{anysize}
\usepackage{graphicx}
\marginsize{1 in}{1 in}{0.5 in}{6 pt}
\title{CMSI 370 Proficiency Check}
\author{Chris Whiting}
%\setlength{\intextsep}{-0.5ex}
\begin{document}
\maketitle

 
\begin{verbatim}
Questions:
1. Name the five usability metrics
2. For each metric, provide:
	(a) A one-line definition of what that metric measures
	(b) An example application for which that metric would have top priority (no need to explain why as 
	      long as the application is sufficiently well-known)
\end{verbatim}

\begin{enumerate}
\item \underline{Learnability} - The time it takes for the user to learn to perform tasks with a system.
\textbf{Ex:} A self-checkout machines at a grocery stores are supposed to speed up the customer checkout process for those only with a few items. The machines should have high learnability, because if it takes the user a long time to figure out how to use it, they may have wasted more time trying to figure it out than actually standing in the normal lines. 
\item\underline{Memorability} - The time it takes to recall a learned application. 
\textbf{Ex:} CPR is something that should have memorability as a priority. Once CPR is learned, the user is not going to know when they have to use it. For this reason, years later when someone is drowning, they should be able to remember how to use CPR to save that person's life.
\item \underline{Efficiency} - How fast it takes for a user to complete a task.
\textbf{Ex:} Google is a well known search engine which allows the user to search for information very efficiently.
\item \underline{Errors} - The rate of errors by the users of a system (user-prone errors).
\textbf{Ex: } A banking site should have very minimal user prone errors because it deals with peoples' money. They require a typing the input of the amount, instead of buttons to specify an amount; to transfer between bank accounts, so that no mistakes are made here, although sacrificing efficiency.
\item\underline{Satisfaction} - The satisfaction the user feels after using a system.
\textbf{Ex:} A video game must provide high satisfaction, so that even if it lacks in other metrics (i.e. learnability), the user will still play the game.
\end{enumerate}

% JD: OK, your new answers work out nicely: +, +  But please proofread better!

\end{document}

