\documentclass[11pt]{article}
\usepackage{anysize}
\usepackage{graphicx}
\marginsize{1 in}{1 in}{0.5 in}{6 pt}
\title{Headmaster Dream Design}
\author{Chris Whiting}
%\setlength{\intextsep}{-0.5ex}
\begin{document}
\maketitle

 
\section{Top Level Design}
My idea for a dream headmaster interface starts with the headmaster login page. When the login page is opened in a browser, it will show the standard "Welcome to Headmaster" title, user id, and password. However, the login screen background will consist of an animated world, the expression "Welcome to Headmaster", and the normal login credential input fields as demonstrated in the mockup of Figure~\ref{homepage}. The mockup was created with elements of the original headmaster combined with an image of a virtual world.

\begin{figure}[h]
\centering
\includegraphics[width=3in]{dreamhome.jpg}
\caption{Animated world - Login Screen Mockup}
\label{homepage}
\end{figure}

As soon as the login button is pressed, the login credentials will fade away and the screen will zoom into a town consisting of a grassy area and one building, representing a school. The town also reads the value of the clock on your computer and outputs the cooresponding time of day where at night a moon would be out and the screen would be darker. Similar to the game Rollar Coaster Tycoon shown in Figure~\ref{RCT}, little animated people will be walking around your town, which signify the students within headmaster.

 Each one is clickable, which brings up that particular students' properties, which will be elaborated later on. At the top left of the page are options which read "Create Event", "Create Grant", "Create Report", and "Logout". Clicking on one of the three create buttons will bring the user to a separate field where a new building must be placed somewhere in the town. This new building is half the size of the school building, regardless of the type of building created. That is, whether you make an event, grant, or report building, they are all half the size of the school building. 

\begin{figure}[h]
\centering
\includegraphics[width=3in]{rollercoastertycoon.jpg}
\caption{Animated world - Rollar Coaster Tycoon}
\label{RCT}
\end{figure}

The event buildings will resemble a big tent, the grants buildings will resemble an office or government building, and the reports buildings will look like ordinary one story houses. Every one of these buildings will be the same color and style, however, their color may be changed. Once one of the smaller buildings is created, a menu similar to the menu from the old headmaster while creating events, grants, or reports. The name of each building is displayed over it and hovering your mouse over any of these buildings will make the name bigger, where these names coorespond to the name of the event, grant, or report.

Double-clicking on any of these buildings will make the screen zoom into the entrance of the respective building and bring up a new view from the ceiling of the building. Many student avatars will be shown corresponding to the students related to that particular event, grant, or report. Students may be added to a building with the Add button at the top left of the screen. When Add is pressed a search bar appears and the user has to enter the name of the student, although a typeahead feature allows for faster additions, because it autocompletes the student name that is being typed. The properties of the event, grant, or report can be edited by pressing an "edit" button at the top left of the screen. Also, next to the edit button is a building icon that brings the user back to the town view and displays "Back to Town View" when hovered over.

 The top of the town webpage has a medium search bar, where the user may search for anything within headmaster, for example students, events, grants, and/or reports. If the user is lazy and has no desire to use the keyboard, a voice command may also be issued by pressing a microphone icon next to the search bar. This feature is similar to Google's seach bar microphone technology. 

\begin{figure}[h]
\centering
\includegraphics[width=3in]{studentavatar.jpg}
\caption{XBOX 360 Avatar Creation}
\label{avatar}
\end{figure}

From the town view, the school building cooresponds to the students tab on the headmaster webpage. When this building is clicked, headmaster will zoom in from the overall town view to the inside of the building. Inside the school building consists of five doors which are again highlighted once the user puts his/her mouse over them. The doors consist of titles which are 'Freshman', 'Sophomores', 'Juniors', 'Seniors', and 'Create'. Similar to the town view above, a search bar is displayed at the top of the school building screen which allows the user to search for an individual student.

Once the user enters any of the four doors, except the Create door, a view of a hallway with virutal people is displayed. The names of each person is displayed above their respective heads and the user may quickly go up and down the hallway via scrolling or the up and down arrow keys. The Create door brings the user to a similar view of editing a student, event, grant, etc., except the avatar is naked. He/she is clothed, however in very plain clothes, which can be customized during and after creation. All student properties start out blank and are displayed to the right of the avatar, as with the Edit button explained below and the Save button.

Once a student is pressed a separate screen is displayed, with the selected student standing and slighting moving around, similar to the XBOX360 avatars shown in Figure ~\ref{avatar}. Also, a menu of student properties is displayed on the right. The student properties consist of Grades, Grants, Information, Food Allergies, etc and an Edit button. At the top of the student properties menu are three small white icons of a tent, an office building, and a house. Clicking on any of these three will add the particular student to the event, grant, and/or report respectively. When the Edit button is pressed, the student properties page changes from non-editable to editable, while the avatar remains in the same place. Back arrows on the upper left of each screen will be displayed bringing the user back to the previous page. However, the user could always press the web browsers back button as well.

Any person avatar in the headmaster system allows for basic customizations. The skin color, hair color, and clothing color can all be customized for each student within headmaster. This allows for variety during the homescreen where all of the avatars are onscreen walking around the virtual town.

\section{Usage Senarios}
The headmaster system is the beginning of a social networking/prowl-like website with many usage senarios. Two of the more common senarios would include creating a new student and creating an event.
\begin{enumerate}
    \item Scenario: Creating a new student
  \begin{enumerate}
        \item Login to headmaster
        \item Click on the school building
        \item Click on the Create Door
        \item Enter desired information and properties to new student.
        \item Save the new student with the save button.
  \end{enumerate}
    \item Scenario: Creating an event
  \begin{enumerate}
        \item Login to headmaster
        \item Click on "Create Event"
        \item Click on anywhere in the town to create a tent building representing the new event.
        \item Enter desired information and properties to the new event.
        \item Press the Add button and add any students to the event.
        \item Save the new event with the save button.
  \end{enumerate}
\end{enumerate}

\section{Design Rationale}
The old headmaster heavily relied on menus for its interaction style, while my dream design is focused on direct manipulation. Menus are still a part of my dream design, however they are only for editing properties of student, events, etc. Another element that is tied with the direct manipulation of my design are its affordances. These are essentially the constraints imposed by an object, or in my design's case, the creation of new buildings. When the user wants to create an event, grant, or report; a new building is created and it is the user's job to pick its placement in the town. This is a drag-and-drop form of direct manipulation where cultural constraints and natural mappings are present. 

A cultural constraint is a communal set of conventions or expectiations, which is essentially "what we're used to". For my design, once a building is created, the building becomes visible at the end of the users mouse icon and moves with the mouse. Once the user clicks, the building will get placed and the properties screen will appear for that particular event, report, etc. Americans know that houses belong in towns next to other buildings, so once the buildings becomes visible on a users' mouse, this should not confuse them, because of this cultural constraint. 

A design has a natural mapping when it holds all of the necessary information to make correct inferences about some aspect of a system, in other words it "just makes sense". The placement of the buildings that the user creates should just make sense to them, but also the avatars of the buildings are logical and make sense. The school building, which holds the students of headmaster, will be unrecognizable at first, but once the user understands that the school building is where all of the students are, it should just make sense. A school is a place where students learn, and because there is only one school building in my design, it just makes sense that if the user wants to look for a student, they should click on the school building. 


\section{Usability Metric Analysis}
By far the strongest usability metric of my headmaster dream design is satisfaction. My design strives to give the user a pleasant experience while using headmaster. From the login screen the user is presented with a visually appealing view of the headmaster virtual world. When the user is navagating through different pages within headmaster, a zoom feature creates a more satisfying experience for the user. When double clicking a spot on Google maps and as the screen zooms in, a screen distortion occurs, which is similar to this headmaster feature.

The weakest metric is learnability, simply because the added satisfaction creates a better user experience, but makes the interface harder to learn. The time it takes for the user to learn an operation may be longer, because of the added satisfaction. When a user attempts to select a current headmaster student, the user must login, navigate to the student building and choose the particular student from the long row of them. However, a medium search bar is provided at the top to match the original headmaster. This helps learnability, because the user can recognize this similar element and already knows how to use it. Therefore, the vastly different interface my dream design suggests is weak in learnability but strong in satisfaction.

\section{Conclusion}
A top level view of my dream design was explained, two usage senarios were given, the design rationale were outlined, and my dream design went through a usability metric analysis. In the top level design, I explained all of the new features of my dream design. Such as the new login page, new look of the headmaster homepage, new way of creating events/students/reports/grants, etc. Not only the functionality, but also the vastly different visual interface my dream design suggested were further explained and illustrated through Figures ~\ref{RCT} and ~\ref{avatar}. Two of the most common usage senarios of headmaster were provided and their solutions were explained based upon my dream design. At the core of my dream design's relevant principles is direct manipulation, which was detailed and explained above. Finally, reasons and explanations were given for the weakest and strongest usability metrics of my headmaster dream design. 

\end{document}

